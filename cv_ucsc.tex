% Don't like 10pt? Try 11pt or 12pt
\documentclass[10pt]{article}

% This is a helpful package that puts math inside length specifications
\usepackage{calc}
\usepackage{url}
\urlstyle{same}
\usepackage{polski}
%\usepackage{enumitem}


% Layout: Puts the section titles on left side of page
\reversemarginpar

%
%         PAPER SIZE, PAGE NUMBER, AND DOCUMENT LAYOUT NOTES:
%
% The next \usepackage line changes the layout for CV style section
% headings as marginal notes. It also sets up the paper size as either
% letter or A4. By default, letter was used. If A4 paper is desired,
% comment out the letterpaper lines and uncomment the a4paper lines.
%
% As you can see, the margin widths and section title widths can be
% easily adjusted.
%
% ALSO: Notice that the includefoot option can be commented OUT in order
% to put the PAGE NUMBER *IN* the bottom margin. This will make the
% effective text area larger.
%
% IF YOU WISH TO REMOVE THE ``of LASTPAGE'' next to each page number,
% see the note about the +LP and -LP lines below. Comment out the +LP
% and uncomment the -LP.
%
% IF YOU WISH TO REMOVE PAGE NUMBERS, be sure that the includefoot line
% is uncommented and ALSO uncomment the \pagestyle{empty} a few lines
% below.
%

%% Use these lines for letter-sized paper
\usepackage[paper=letterpaper,
            %includefoot, % Uncomment to put page number above margin
            marginparwidth=1.2in,     % Length of section titles
            marginparsep=.05in,       % Space between titles and text
            margin=1in,               % 1 inch margins
            includemp]{geometry}

%% More layout: Get rid of indenting throughout entire document
\setlength{\parindent}{0in}

%% This gives us fun enumeration environments. compactenum will be nice.
\usepackage{paralist}

%% Reference the last page in the page number
%
% NOTE: comment the +LP line and uncomment the -LP line to have page
%       numbers without the ``of ##'' last page reference)
%
% NOTE: uncomment the \pagestyle{empty} line to get rid of all page
%       numbers (make sure includefoot is commented out above)
%
\usepackage{fancyhdr,lastpage}
\pagestyle{fancy}
%\pagestyle{empty}      % Uncomment this to get rid of page numbers
\fancyhf{}\renewcommand{\headrulewidth}{0pt}
\fancyfootoffset{\marginparsep+\marginparwidth}
\newlength{\footpageshift}
\setlength{\footpageshift}
          {0.5\textwidth+0.5\marginparsep+0.5\marginparwidth-2in}
\lfoot{\hspace{\footpageshift}%
       \parbox{4in}{\, \hfill %
                    \arabic{page} of \protect\pageref*{LastPage} % +LP
%                    \arabic{page}                               % -LP
                    \hfill \,}}

% Finally, give us PDF bookmarks
\usepackage{color,hyperref}
\definecolor{darkblue}{rgb}{0.0,0.0,0.3}
\hypersetup{colorlinks,breaklinks,
            linkcolor=darkblue,urlcolor=darkblue,
            anchorcolor=darkblue,citecolor=darkblue}

\renewcommand{\labelitemi}{--}

% The title (name) with a horizontal rule under it
%
% Usage: \makeheading{name}
%
% Place at top of document. It should be the first thing.
\newcommand{\makeheading}[1]%
        {\hspace*{-\marginparsep minus \marginparwidth}%
         \begin{minipage}[t]{\textwidth+\marginparwidth+\marginparsep}%
                {\large \bfseries #1}\\[-0.15\baselineskip]%
                 \rule{\columnwidth}{1pt}%
         \end{minipage}}

% The section headings
%
% Usage: \section{section name}
%
% Follow this section IMMEDIATELY with the first line of the section
% text. Do not put whitespace in between. That is, do this:
%
%       \section{My Information}
%       Here is my information.
%
% and NOT this:
%
%       \section{My Information}
%
%       Here is my information.
%
% Otherwise the top of the section header will not line up with the top
% of the section. Of course, using a single comment character (%) on
% empty lines allows for the function of the first example with the
% readability of the second example.
\renewcommand{\section}[2]%
        {\pagebreak[2]\vspace{1.3\baselineskip}%
         \phantomsection\addcontentsline{toc}{section}{#1}%
         \hspace{0in}%
         \marginpar{
         \raggedright \scshape #1}#2}

% An itemize-style list with lots of space between items
\newenvironment{outerlist}[1][\enskip\textbullet]%
        {\begin{enumerate}[#1]}{\end{enumerate}%
         \vspace{-.6\baselineskip}}

% An itemize-style list with little space between items
\newenvironment{innerlist}[1][\enskip\textbullet]%
        {\begin{compactenum}[#1]}{\end{compactenum}}

% To add some paragraph space between lines.
% This also tells LaTeX to preferably break a page on one of these gaps
% if there is a needed pagebreak nearby.
\newcommand{\blankline}{\quad\pagebreak[2]}

\newcommand{\OG}{\textbf{O. Gallo}}

\begin{document}
\makeheading{Orazio Gallo, Ph.D.\\ \small{Senior Research Scientist, NVIDIA Research}}

\section{Contact Information}
%\rcollength is the width of the right column of the table
%       (adjust it to your liking; default is 1.85in).
%
%
\newlength{\rcollength}\setlength{\rcollength}{2.1in}%{2.9in} <- this number if you add back the NVIDIA website
\begin{tabular}[t]{@{}p{\textwidth-\rcollength-0.1in}rp{\rcollength}}
Website:& \href{http://www.soe.ucsc.edu/~orazio}{\url{http://www.soe.ucsc.edu/~orazio}}\\
%& \href{https://research.nvidia.com/users/orazio-gallo}{\url{https://research.nvidia.com/users/orazio-gallo}}\\
& \href{https://scholar.google.com/citations?user=6UHjQQYAAAAJ&hl=en&oi=ao}{Google Scholar}\\
E-mail: & \href{mailto:ogallo@nvidia.com}{\url{ogallo@nvidia.com}}\\
%Cell: &  +1-415-217-9592
\end{tabular}
%
%\newlength{\rcollength}\setlength{\rcollength}{3in}
%\begin{tabular}[t]{@{}p{\textwidth-\rcollength-0.1in}rp{\rcollength}}
%Websites: & \href{http://www.soe.ucsc.edu/~orazio}{\url{www.soe.ucsc.edu/~orazio}}\\
% & \href{https://research.nvidia.com/users/orazio-gallo}{\url{https://research.nvidia.com/users/orazio-gallo}}\\
%E-mail:   & \href{mailto:orazio@soe.ucsc.edu}{\url{orazio@soe.ucsc.edu}}%\\
%%Cell: &  +1-415-217-9592
%\end{tabular}

\section{Research Interests}
Mobile Visual Computing, Computational Photography, Computer Vision, and Visual Perception.

Modern mobile devices, such as smartphones and tablets, offer unprecedented computational power, in part thanks to the graphics processing units (GPUs) with which they are generally equipped. I am interested in leveraging this power to explore novel paradigms for visual computing in the wild.

\section{Professional Experience}
\textbf{\emph{Senior Research Scientist in the Mobile Visual Computing Team at NVIDIA Research.}} (2011---present)\\
I worked on several mobile visual computing projects, ranging from computer vision to computational photography. These include studies on several aspects of stack-based photography, novel hardware designs for cameras, alternatives to traditional image processing pipelines, new paradigms for capturing and consuming photos, and displays that leverage human perception mechanisms to provide a more immersive visual experience.\newline

\textbf{\emph{Computer Vision Adviser for dfxMACHINA.}} (2015---present)\\
dfxMACHINA is a stealth-mode startup that aims at using computer vision to predict and troubleshoot pitfalls of pipelined manufacturing processes. I am helping define the computer vision strategy of the company.\newline

%I am interested in several aspects of the high-dynamic-range (HDR) imaging pipeline---from capture, to registration, to display. I am also looking at alternative image processing pipelines, new camera designs, and novel paradigms for the capture and consumption of images, including the study of the visual perception mechanisms involved in the process.\newline

\textbf{\emph{Research Intern at the Nokia Research Center, Palo Alto.}} (Summer 2008, Winter 2010, and Winter 2011)\\
I worked on several mobile computational photography algorithms, with a focus on high-dynamic-range (HDR) photography, from novel metering algorithms, to strategies to create ghost-free HDR images in the common case of non-static scenes.\newline

\textbf{\emph{Research Intern at Canesta, Inc.}} (Summer 2007)\\
I worked on time-of-flight data processing and, specifically, on robust algorithms to fit planes to range data in the presence of multiple surfaces and noise.\newline

\textbf{\emph{Research Assistant at the Smith-Kettlewell Eye Research Institute.}} (November 2004---June 2006)\\I worked on the  development of a new imaging technique, the Gold Bead Tissue Markers project. I was in charge of the design and implementation of image processing, matching, and 3-D reconstruction algorithms for the imaging prototype.%\newline

\section{Education}
%!TEX root = cv.tex
\vspace{-7mm}
\begin{itemize}[-]
\item Ph.D. in Computer Engineering at the University of California, Santa Cruz, 2011. Dissertation: \emph{``Applications of mobile vision to information access and computational imaging.''}

\item  ``Laurea'' degree (equivalent to a M.S.) in Biomedical Engineering at ``Politecnico di Milano,'' Italy, 2004.\\
%I carried out my thesis research while visiting the Telerobotics and Neurology Laboratory of Prof. Lawrence W. Stark at UC Berkeley, under the supervision of Prof. Stark and Dr. Claudio M. Privitera.
Master's thesis: \emph{``Prediction of the Regions-Of-Interest of images: algorithms combinations evaluated with scanpath acquisitions.''}\\
I carried out this research while visiting the Telerobotics and Neurology Laboratory of Prof. Lawrence W. Stark at the University of California, Berkeley.%, under the supervision of Prof. Stark and Dr. Claudio M. Privitera.

\end{itemize}


%-- Ph.D. in Computer Engineering at the University of California, Santa Cruz, 2011.
-- ``Laurea'' degree (equivalent to a M.S.) in Biomedical Engineering at ``Politecnico di Milano,'' Italy, 2004.

\newpage
\section{Professional Activities}
\textbf{\emph{Associate Editor:}} Signal Processing: Image Communication (2015---present)\\

\textbf{\emph{Program Committees:}}\\
SIGGRAPH Asia Technical Briefs Committee (2015)\\
SIGGRAPH Asia Posters Committee (2015)\\
ICCV (2015)\\
IEEE CVPR (2015, 2016)\\
IEEE ICCP (2015, 2016)\\
Eurographics State-Of-The-Art (STAR) Reports (2015)\\
Embedded Vision Workshop (2012, 2013, 2014, 2015)\\
Workshop on Photographic Aesthetics and Non-Photorealistic Rendering (2013)\\

\textbf{\emph{Reviewer For:}}\\
SIGGRAPH, SIGGRAPH ASIA, Eurographics, IEEE CVPR, ICCV, IEEE WACV, Eurographics Symposium on Rendering.\\
IEEE Transactions on Pattern Analysis and Machine Intelligence, IEEE Transactions on Multimedia, 
IEEE Transactions on Signal Processing,
IEEE Transactions on Systems Man and Cybernetics,
ACM Transactions on Applied Perception,
IEEE Sensors,
Computer Vision and Image Understanding,
International Journal of Computer Vision,
Journal of Imaging Science  \& Technology, 
The Visual Computer,
Computers \& Graphics,
Applied Optics.


%\section{Programming Skills} 
%MatLab and C/C++.\newline
\section{Invited Talks} \emph{Stack-based Photography and Advanced Processing in Embedded Camera Systems}, Embedded Vision Summit, Santa Clara, 2016.


\section{Teaching Experience} Teaching Assistant for the class \emph{CMPE80A: Universal Access: Disability, Technology, And Society}, University of California, Santa Cruz (Instructor Prof. Roberto Manduchi).



%\newpage
\section{Book Chapters} 
\OG{} and P. Sen, \emph{Stack-Based Approaches to HDR Capture and Reconstruction} (in ``High Dynamic Range Video--–From Acquisition to Display and Applications'', Academic Press, in press).

\section{Select Publications}
D.E. Jacobs, \OG{}, E.A. Cooper, K. Pulli, M. Levoy, \emph{Simulating the Visual Experience of Very Bright and Very Dark Scenes}. ACM Transactions on Graphics, 2015.\\

F. Heide, M. Steinberger, Y. Tsai, M. Rouf, D. Paj\k{a}k, D. Reddy, \OG{}, J. Liu, W. Heidrich, K. Egiazarian, J. Kautz, K. Pulli, \emph{FlexISP: A Flexible Camera Image Processing Framework} ACM SIGGRAPH Asia, 2014.\\

J. Hu, \OG{}, K. Pulli, X. Sun, \emph{HDR Deghosting: How to Deal with Saturation?} IEEE Conference on Computer Vision and Pattern Recognition (CVPR), 2013.\newline

\OG{} and R. Manduchi, \emph{Reading 1-D Barcodes with Mobile Phones Using Deformable Templates}, IEEE Transactions on Pattern Analysis and Machine Intelligence (PAMI), 2011.\newline

\OG{}, N. Gelfand, W. Chen, M. Tico, and K. Pulli, \emph{Artifact-free High Dynamic Range Photography},  International Conference of Computational Photography (ICCP), 2009.\newline


\section{Other Publications}
H. Zhao, \OG{}, I. Frosio, J. Kautz, \emph{Is $\ell_2$ a Good Loss Function for Neural Networks for Image Processing?}, arXiv:1511.08861, 2015, (submitted to CVPR 2016).\\

\OG{}, A. Troccoli, J. Hu, K. Pulli, J. Kautz, \emph{Locally Non-rigid Registration for Mobile {HDR} Photography}, Embedded Vision Workshop, CVPR, 2015.\\

\OG{}, I. Frosio, L. Gasparini, K. Pulli, M. Gottardi, \emph{Retrieving Gray-Level Information from a Binary sensor and its application to Gesture Detection}, Embedded Vision Workshop, CVPR, 2015.\\

D.E. Jacobs, \OG{}, K. Pulli, \emph{Dynamic Image Stacks},  IEEE Workshop on Mobile Vision, CVPR, 2014.\\

J. Hu, \OG{}, and K. Pulli, \emph{Exposure Stacks of Live Scenes with Hand-held Cameras}, European Conference of Computer Vision (ECCV), 2012.\newline

\OG{}, M. Tico, R. Manduchi, N. Gelfand, and K. Pulli, \emph{Metering for Exposure Stacks}, Eurographics, 2012.\newline

\OG{}, R. Manduchi, and A. Rafii, \emph{CC-RANSAC: Fitting Planes in the Presence of Multiple Surfaces in Range Data}, Pattern Recognition Letters, 2011.\newline

J. Davis, J. Arderiu, H. Lin, Z. Nevins, S. Schuon, \OG{}, and M. Yang, \emph{The HPU}, Workshop on Advancing Computer Vision with Human in the Loop, CVPR, 2010.\newline

\OG{} and R. Manduchi, \emph{Reading Challenging Barcodes with Cameras}, Workshop on Applications of Computer Vision (WACV), 2009.\newline

\OG{}, S.M. Arteaga, and J.E. Davis, \emph{A camera-based pointing
interface for mobile devices}, International Conference of Image Processing (ICIP), 2008.\newline

\OG{}, R. Manduchi, and A. Rafii, \emph{Robust curb and ramp detection for
safe parking using the Canesta TOF camera},  Workshop on Time-of-Flight-based Computer Vision, CVPR, 2008.\newline

J.M. Miller, E.A. Rossi, M. Wiesmair, D.E. Alexander, and \OG{},\label{paper:GBTM}
\emph{Stability of gold bead tissue markers}. Journal of Vision (JOV), 2006.\newline

C.M. Privitera, \OG{}, G. Grimoldi, T. Fujita, and L.W. Stark,
\emph{Combining conspicuity maps for hROIs prediction}, Workshop on
Attention and Performance in Computational Vision, ECCV, 2004.\newline

%\newpage

\section{Patents}
D. S. Paj\k{a}k, F. Heide, N.D. Reddy, M. Rouf, J. Kautz, K. Pulli, \OG{}, \emph{Unified Optimization Method For End-to-end Camera Image Processing For Translating a Sensor-captured Image to a Display Image}, pending.\\

\OG{}, K. Pulli, J. Hu, \emph{System, Method, and Computer Program for performing Fast, Non-Rigid Registration, for HDR Image Stacks}, pending.\\

\OG{}, K. Pulli, J. Hu, \emph{Methods and Apparatus for Registering Image Stacks}, 2014.\\

K. Pulli, \OG{}, D.E. Jacobs, \emph{Interaction with and Display of Photographic Images in an Image Stack}, 2014.\\

\OG{} and R. Manduchi, \emph{Image-based barcode reader}, 2012.\newline

\section{Awards}
Chancellor's Fellowship for outstanding incoming graduate students, University of California, Santa Cruz, 2006.

\section{Other Interests}
I love taking pictures, in particular of the {\em street photography} kind. My username on Flickr is 
\href{http://www.flickr.com/photos/16891554@N03/}{Paparazio}.\\


I spend a (large) chunk of my (little) free time training and playing beach volleyball at Main Beach in Santa Cruz.

%\section{References}
%Available upon request.

\pdfinfo{
   /Author (Orazio Gallo)
   /Title  (Orazio's Resume)
   /Keywords (Computational Photography, High Dynamic Range, HDR, metering, Computer Vision, Mobile Vision, mobile computing)
}

\end{document}

%%%%%%%%%%%%%%%%%%%%%%%%%% End CV Document %%%%%%%%%%%%%%%%%%%%%%%%%%%%%
